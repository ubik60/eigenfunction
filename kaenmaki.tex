\documentclass[12pt]{scrartcl}

\usepackage{mydefs}

\addtolength{\parskip}{12pt}
\setlength{\parindent}{0pt}

\usepackage{lmodern}

\title{Käenmäki example}
\begin{document}

\section*{Käenmäki example}
This example provides a $g$-measure $\mu$ such that the ratio
\[
  \frac{\mu([x_1,x_2,\dots,x_m,y_1,y_2,\dots,y_m])}
  {\mu([x_1,x_2,\dots,x_m])\cdot \mu([y_1,y_2,\dots,y_m])}
\]
is uniformly bounded from above but not from below. A restatement says that the
likelihood ratios between any conditioning of the measure and measure itself,
i.e.  
\[
\frac{\mu([x_1,x_2,\dots,x_m] \mid y)}{\mu([x_1,x_2,\dots,x_m])} = 
\frac{\mu([x_1,x_2,\dots,x_m] \mid y)}{\int\mu([x_1,x_2,\dots,x_m \mid z])\d\mu(z)},
\]
are bounded from above but not from below. 

Let $X=\{0,1\}^S$, $S=\{0,1,2,\dots\}$, and let $\mu\in\CM(X)$ be the unique
Hofbauer-type $g$-measure with
\[ g(0x) = e^{-s_{W(x)}},\quad g(1x) = 1 - e^{-s_{W(x)}} \]
where $W(x)$ is the number of initial zeros in $x\in X$ and 
\[
  s_k = 2-\frac1{1+k}.
\]
Note that $\mu$ is sandwiched (in stochastic dominance ordering) between two iid
Bernouilli sequences with $p=1-e^{-1}$ and $p=1-e^{-2}$.

For the Hofbauer-type measure a likelihood ratio only
depends on the span of the zero-sequence, i.e., if we set $V=W(y)$ then 
\[
  \frac{\mu([x_1,x_2,\dots,x_m]| y)}{\mu([x_1,x_2,\dots,x_m])} 
  = \frac{\mu\left([0_{k}] \mid [0_{V}, 1]\right)}{\mu([0]_k)}
\]
where $0_k$ denotes a
sequence of $k$ zeros and where $k$ is given by the condition that either $k=m$
or else
\[
  x_{m-k}=1,\quad x_{m-k+1}=x_{m-k+2}=\dots=x_{m}=0.
\]
Hence, it is enough to
consider the case $(x_1,\dots,x_m)=0_m$.

Let $V=W(y)$ be the waiting time for the sequence $y$. Note that the ratio
\begin{align*}
  \frac{\mu(0_m)}{\mu(0_m\mid y)}
  &=
    \int \exp\left\{ (s_V-s_W) + (s_{V+1}-s_{W+1}) + \dots + (s_{V+m-1}-s_{W+m-1}) 
    \right\} \d\mu(z)
  \\
  &=
    \int \exp\left\{ \sum_{k=W+1}^{W+m} \frac 1k - \sum_{k=V+1}^{V+m}\frac 1k \right\}
    \d\mu(z),
\end{align*}
where $W=W(z)$. 

Since $\sum_{k=1}^N \frac 1k = \ln N + \gamma +\ordo1$ as
$N\to\infty$, we can deduce that
\begin{align*}
  \frac{\mu([0_m])}{\mu([0_m]\mid y)}
  &= 
  \int \exp\left\{ \sum_{k=W+1}^{W+m} \frac 1k - \sum_{k=V+1}^{V+m}\frac 1k \right\} \d\mu(z) \\
  &\simeq 
  \frac{V+1}{V/m+1}\cdot \int \frac{W/m+1}{W+1} \d\mu(z).
\end{align*}
where $f \simeq g$ here means that $|f/g|$ is bounded away from $0$ and $\infty$
as for all $m$.

Since $W(z)$ is sandwiched between two geometric distributions with $
p=1-e^{-2}$ and $p=1-e^{-1}$, it is clear that the integral 
\[
  I=\int \frac{W/m+1}{W+1} \d\mu(z) = \frac 1m
  + (1-\frac1m) \cdot \E\left(\frac{1}{W+1}\right)
\]
satisfies 
\[
 (1-e^{-1}) <  I < 1. 
\]

Thus, we obtain the following expression for the likelihood ratios 
\[
  \frac{\mu([0_m]\mid y)}{\mu([0_m])} \simeq \frac1I \cdot \frac{V/m+1}{V+1}.
\] 
The right hand side is less than $1/I$. Hence
the likelihood ratios are bounded from above. On the other hand, if we let
$V\to\infty$ then
$$  \frac{\mu([0_m]\mid y)}{\mu([0_m])} \to \frac1I \cdot \OrdoTheta{\frac 1m} $$
and the likelihood-ratios are thus not bounded away from zero since $m$ is
arbitrary.

\end{document}

