\documentclass[11pt, a4paper]{amsart}
\usepackage[textsize=footnotesize,textwidth=3cm]{todonotes}
\usepackage[parfill]{parskip} % Begin paragraphs with an empty line rather than an indent


\usepackage{amsthm}
\newtheorem{thm}{Theorem}
\newtheorem{theorem}[thm]{Theorem}
\newtheorem{conjecture}[thm]{Conjecture}
\newtheorem{cor}[thm]{Corollary}
\newtheorem{corollary}[thm]{Corollary}
\newtheorem{lem}[thm]{Lemma}
\newtheorem{lemma}[thm]{Lemma}
\newtheorem{prop}[thm]{Proposition}
\newtheorem{proposition}[thm]{Proposition}

\newtheorem{axiom}{Axiom}
\newtheorem{claim}{Claim}
\theoremstyle{definition}
\newtheorem{defn}{Definition}
\newtheorem{definition}{Definition}
\newtheorem{ex}{Example}
\newtheorem{example}{Example}
\theoremstyle{remark}
\newtheorem{notation}{Notation}
\newtheorem{remark}{Remark}
\newtheorem{rem}{Remark}

\PassOptionsToPackage{override}{xcolor} % Beamer option
\usepackage[utf8]{inputenc}
\usepackage{hyperref}
\usepackage{amsfonts,amssymb,amsmath}
\usepackage{dsfont}
\usepackage[mathscr]{euscript}
\usepackage{enumerate,xspace,threeparttable}
\usepackage{graphicx}
\usepackage{verbatim}

%\providecommand{\ie}{i.e.\ }
\providecommand{\qr}{\eqref}
\renewcommand{\d}{\,d}
\providecommand{\dd}[2]{\dfrac{d#1}{d#2}}
\providecommand{\qtext}[1]{\quad\text{#1 }\quad}

\providecommand{\RR}{\mathbb{R}}
\providecommand{\CC}{\mathbb{C}}
\providecommand{\ZZ}{\mathbb{Z}}
\providecommand{\QQ}{\mathbb{Q}}
\providecommand{\NN}{\mathbb{N}}
\providecommand{\CF}{\mathscr{F}}
\providecommand{\CB}{\mathscr{B}}
\providecommand{\CA}{\mathscr{A}}
\providecommand{\CM}{\mathscr{M}}
\providecommand{\CT}{\mathscr{T}}

\providecommand{\mfrak}{\mathfrak}
\providecommand{\mscr}{\mathscr}
\providecommand{\mc}{\mathcal}
\providecommand{\mb}{\mathbf}
\providecommand{\bs}{\boldsymbol}
\providecommand{\mbb}{\mathbb}
\providecommand{\ms}{\mathsf}
\providecommand{\vv}[2]{\ensuremath{\overrightarrow{#1#2}}} % vector
\providecommand{\opn}{\operatorname}
\providecommand{\ol}{\overline}
\def\ii#1{^{(#1)}}
\def\pp#1{\left(#1\right)}

\renewcommand{\P}{\mathsf{P}}
\renewcommand{\Pr}[1]{\P\pp{#1}}
\providecommand{\E}{\mathsf{E}}
\providecommand{\Ex}[1]{\E\pp{#1}}

\providecommand{\Var}{\opn{Var}}
\providecommand{\var}{\opn{var}} 
\providecommand{\Cov}{\opn{Cov}}
\providecommand{\msf}{\mathsf}
\providecommand{\ett}{\mathsf{1}}

\providecommand{\e}{\epsilon}
\providecommand{\tl}{\tilde}
\providecommand{\g}{\gamma}
\providecommand{\w}{\omega}
\providecommand{\scp}[2]{\ensuremath{\left\langle#1,#2\right\rangle}}

\providecommand{\bmat}[1]{\begin{bmatrix} #1 \end{bmatrix}}
\providecommand{\ds}{\displaystyle}
\renewcommand{\L}{\mathscr{L}}

\def\X{X}
\def\T{\opn{\msf{T}}}
\def\F{\mscr F}

\def\scp#1#2{\left\langle #1 , #2 \right\rangle}

\title[Existence of continuous eigenfunction]{Existence of continuous eigenfunctions for the Dyson model in the critical phase}

\author{Anders Johansson, Anders \"Oberg, and Mark Pollicott}
\date{}
\begin{document}

\maketitle
\begin{abstract}
We prove that there exists a continuous eigenfunction of the transfer operator defined by potentials for the so-called Dyson model for all inverse critical temperatures that are strictly less than the critical inverse temperature. This includes all cases when the potential does not have summable variations, the classical condition that ensures the existence of a continuous eigenfunction of the transfer operator. As a consequence the inverse critical temperatures for the one-sided and the two-sided models are the same. We use the random cluster model for our method of proof, so the results also hold in this wider context. In particular, we base our conclusions of this paper by recent results of Duminil-Copin, Garban, and Tassion \cite{duminil}.
\end{abstract}
\def\h{h}


\section{Introduction}\noindent

It is well-known \cite{walters1} that there exists a continuous and strictly positive
eigenfunction $h$ for any transfer operator defined on a symbolic shift space
with a finite number of symbols such that the potential has summable variations.
Here we prove the existence of a continuous eigenfunction for the important special class
of Dyson potentials up to the critical phase, when the potential does not
satisfy the condition of summable variations. We stress that it is the continuity that is the main difficulty. 
To see that there is a measurable eigenfunction follows from a simple application of the martingale 
convergence theorem.

More precisely, let $T$ be the left shift on the space $X_+=S^{{\mathbb Z}_+}$,
where $S$ is a finite set. Define a transfer operator ${\mathcal L}$ on
continuous functions $f$ by
\begin{equation}\label{trans} {\mathcal L} f(x)= \sum_{y\in T^{-1}x}
  e^{\phi(y)}f(y),
\end{equation}
where $\phi$ is a continuous potential. Since $X_+$ is a compact space, it
follows automatically from the Schauder--Tychonoff theorem that the dual
${\mathcal L}^*$, restricted to the probability measures, has an eigenmeasure
$\nu$: ${\mathcal L}^* \mu=\lambda \nu$, for some $\lambda>0$. The existence of
a continuous eigenfunction $h$ such that ${\mathcal L}h=\lambda h$, $\lambda>0$,
is however not automatic. 

If one assumes summable variations of $\phi$,
\begin{equation}\label{sum}
  \sum_{n=1}^\infty \var_n (\phi)<\infty,
\end{equation}
where $\var_n(\phi)=\sup_{x\sim_n y}|\phi(x)-\phi(y)|$ ($x\sim_n y$ means that
$x$ and $y$ coincide in the first $n$ entries), then existence of a continuous
eigenfunction follows from the typical ``cone-argument'' used in {\em e.g.},
Walters \cite{walters1}, which to date is the only
known method for providing the existence of a continuous eigenfunction.

If $\mu$ is the equilibrium measure (translation invariant Gibbs measure) for
the continuous potential $\psi:X\to X$, where $X=S^{\mathbb Z}$, then (with
slight abuse of notation) $\mu$ is recovered on $X_+$ as an eigenmeasure for the
probability potential
\begin{equation}\label{g}
  g(x)= \frac{h(x) e^{\phi(y)}}{\lambda h(Tx)}.
\end{equation} 
More precisely, for a transfer operator ${\mathcal L}_g$, defined on continuous functions $f$ by
\begin{equation} {\mathcal L}_g f(x)=\sum_{y\in T^{-1}x} g(y) f(y)
\end{equation}
we have ${\mathcal L}_g^*\mu=\mu$, and also the invariance $\mu\circ
T^{-1}=\mu$. In \cite{johob} it was proved that uniqueness of such $g$-measures (in the terminology of Keane \cite{keane}) follows if
we have
$$\sum_{n=1}^\infty (\var_n g)^2<\infty,$$
and more recently Berger et al.\  \cite{berger2} proved that for such {\em Doeblin measures} uniqueness follows if
$\var_n \log g <2/\sqrt{n}$ (see also \cite{johob3} for a slightly stronger condition). This is in contrast to Dyson's counterexample in \cite{dyson} where it is shown that there are examples of multiple equilibrium measures for $\phi$ whenever 
$$\sum_{n=1}^\infty (\var_n \phi)^{1+\epsilon}<\infty,$$
where $\epsilon>0$. Dyson's example is for a two-sided model, but we showed in \cite{johob4} that the inverse critical temperature $\beta_c^+$ satisfies $\beta_c^+\leq 8\beta_c$, where  $\beta_c$ is the inverse critical temperature for the two-sided model, and this show that Dyson's example of multiple equilibrium measures can be formulated for a one-sided model as above.

Since we have the ``translation'' via \eqref{g} between the case for general potentials and for Doeblin measures, we may guess that the existence of a continuous eigenfunction cannot be moved very far away from the summability of variations condition for a potential.

One possibility for the existence of a continuous eigenfunction in general could be the Berbee condition.
Berbee proved (\cite{berbee89}) that there exists a
unique equilibrium measure whenever
\begin{equation}\label{berbee}
  \sum_{n=1}^\infty e^{-r_1-r_2-\cdots-r_n}=\infty,    
\end{equation}
where $r_n=\var_n \log \phi$. Berbee's condition
gives uniqueness for both the two-sided and
one-sided potentials

Here, however, we study only the Dyson potential: Fix $\alpha>1$ and $\beta>0$. Let the one-point potential $\phi$ be given by
$$\phi(x_0, x_1,\ldots)=x_0\cdot \beta \sum_{j=1}^\infty \frac{x_j}{j^\alpha},$$
and define the one-sided and two-sided 
Dyson potentials, $\phi:\X_+\to \RR$ and $\psi:\X\to\RR$, respectively, as 
$$
\phi(x)=\sum_{k=0}^\infty x_k \theta (T^k x) \quad
\psi(x)=\sum_{k=-\infty}^\infty x_k \theta(T^k x).
$$
Let $\mu$ and $\nu$ be the Gibbs measures on $\CM(\X)$ and $\CM(\X_+)$, corresponding to $\psi$ and $\phi$, respectively.  

We then have for $\alpha>2$
$$\mu\vert_{{\mathcal F}_{[0,\infty)}}= h\nu,$$
where $\mu$ is the two-sided translation invariant Gibbs measure (the
equilibrium measure) and where ${\mathcal L}^*\nu=\lambda \nu$, and $h>0$ is a
H\"older continuous eigenfunction. We are interested in the boundary case
$\alpha=2$, when there exists a unique equilibrium measure for $\psi$ for
$\beta<\beta_c$ \cite{ACCN}. In this case the summable variations condition is not satisfied
for neither $\psi$ nor $\phi$; hence we may have multiple eigenmeasures for
${\mathcal L}^*$. In this context we have $\var_n(\phi)=O(\frac{1}{n})$, but as we noted earlier, in the general setting above, is not clear that there exists a continuous eigenfunction even in the cases we have a unique
equilibrium measure

We prove that if the inverse critical temperature $\beta$ is strictly smaller than the critical inverse temperature $\beta_c$, which will be seen to be the same for the two-sided and one-sided models, then we have a continuous eigenfunction of ${\mathcal L}$.

\begin{thm}\label{main} Let $\mu$ be the Gibbs equilibrium measure with respect
to the Dyson potential $\phi$ and let $\nu$ be the one-sided Gibbs measure,
i.e., ${\mathcal L}^*\nu=\lambda \nu$. Define
$$h_n(x)=\frac{\mu[x_0,\ldots x_n]}{\nu[x_0,\ldots, x_n]},$$
and consider the measurable function $h(x)=\lim_{n\to \infty}h_n(x)$ (that exists by virtue of the martingale convergence theorem). If $\beta<\beta_c$, then $h$ is a continuous function on $X_+$, and which is also
an eigenfunction of ${\mathcal L}$.
\end{thm}

We conjecture that there exists a continuous eigenfunction for a potential
$\phi$ that satisfies Berbee's condition \eqref{berbee}.

The next result is a corollary of Theorem 1:

\begin{thm}
The critical $\beta_c$ for the two-sided model is the same as for the one-sided model, i.e., 
$\beta_c^+=\beta_c$.
\end{thm}

This improves our result of Theorem 1 in \cite{johob4} that $\beta_c^+\leq 8\beta_c$.
\section{The one-dimensional random cluster model and the Ising--Dyson model}

For a finite graph, let $\w(G)$ denote the number of connected components
(``clusters'') in the graph $G$. For simple graphs $G\subset \binom V2$ on an
countably infinite set $V$ of vertices, we consider the number of clusters
$\w(G)$ as a \emph{potential}. This means that the difference
$\Delta\w(G,F) = \w(G)-\w(F)$ is defined for any two graphs $F$ and $G$ that
coincide outside a \emph{finite} subset $\Lambda\subset \binom V2$.

A random graph $G\sim\alpha$ on a set of vertices $V$ is a probability
distribution $\alpha$ on the set $\{0,1\}^{\binom V2}$. The random cluster
models $\mscr R(V,p)$, we consider are specified by a set of vertices $V$ and a
weight function $p$, $ij\mapsto p(ij)\in[0,1]$, defined on the set of pairs
$ij\in \binom V2$. The model $\mscr R(V,p)$ is a Gibbs distribution on random
graphs, i.e.\ configurations i $\{0,1\}^{\binom V2}$, such that a

In the one-dimensional Ising--Dyson model we let $V=\ZZ$ (or $V=V_+=\ZZ_{\ge0}$ for
the one-sided case) and for $i,j\in V$ let
\begin{equation}\label{eq:Jdef}
  J({ij}) = \frac \beta{|i-j|^\alpha}.
\end{equation}
We will consider the case $\alpha=2$ and $0<\beta <\beta_c$. For $\beta>0$, let
\(\g\sim \eta \) be the Bernoulli random graph $\eta \in \mscr G(V)$ with
$$
\P(ij \in \g)=1-e^{-\beta J({ij})}.
$$

The extended \emph{random cluster model} can be obtained by considering the
product measure $\d\eta(\g) \otimes \d\ett(x)$ between an independent
Bernoulli distributed $\g\sim\eta_p$ random graph on $\mscr G(V)$ and the
uniform measure $x\sim\ett$ on the spin sequences $x\in\{-1,+1\}^{V}$. The
extended random cluster model $\d\mu(x,\g)$ is the joint distribution of $\g$
and $x$ obtained by conditioning on the on the event that $x$ and $\g$ are
compatible: That is, the event $C(x,\g)$ that no spins $x_i=+1$ and $x_j=-1$ in
$x$ are connected by a path (edge) in $\g$. One obtains the
Ising model $\d\mu(x)$ and the random cluster model $\d\mu(\g)$ as the marginal
distributions of $x$ and $\g$, respectively.

We can also introduce the random cluster model $\mu$ as the Gibbs measure on
$\{0,1\}^{\binom V2}$
$$ \d\mu = 2^{\w(\g)} \ltimes \d\eta(\g), $$
where $\w(\g)$ is the potential counting then number of connected components
(``clusters'') in $\g$. For the values of $\beta$ we consider the Gibbs measure
is unique. Thus can we parameterise the random clusters models as
$\mu = \mscr R(V,J)$ where $J(ij)$ is a given weighting on $\binom V2$ such as
\eqref{eq:Jdef}.

We observe the  difference between the one-sided random cluster model $\nu = \mscr R(V_+,J)$ and
the usual two-sided model $\mu = \mscr R(V,J)$. We will use that a configuration
$\g$ can be factored as $\g = (\g_+, \e, \g_-)$, where $\g_-$ is the induced
graph $\g[V_-]$ on vertices $-j\in V_-=\ZZ_{<0}$ and $\g_+=\g[V_+]$ is the graph
induced on vertices $i\in V_+=\ZZ_{\ge0}$. The graph $\e=\g\cap E(V_+,V-)$
consists of edges $ji$, $i\ge0$ and $j\ge 1$, connecting vertices $-j\in V_-$
with vertices $i\in V_+$. Note that we often use positive indices $i,j$, $i\ge0$
and $j>0$, as labels for edges in $\e$. Thus $J(ij)=\beta/(i+j)^\alpha$ with
this labelling.

Let
$$
\tilde \eta(\epsilon)= 2^{-|\epsilon|} \ltimes \eta (\epsilon).
$$
Note that both $\eta(\cdot)$ and $\tilde\eta (\cdot)$ are Bernoulli measures.
Let also
$$
d\tilde \nu_n(\gamma) =
\frac{d\nu(\gamma_-)\otimes \tilde \eta(\epsilon)\otimes \nu(\gamma_+)}{\nu(B([x]_n,\gamma_+))}
$$

Let $R_n$ be the number of correcting edges, i.e.,
$$
R_n=\# \{ij\in \epsilon: \omega (\gamma_{< ij}+ij)=\omega (\gamma_{ij}\},\quad j>n.
$$
Let $B_n(\g)=C([x]_n,\gamma)$ be the indicator of the event that cylinder $[x]_n$ is compatible with graph $\g=(\g_-,\e,\g_+)$.
Let also $B'_n(\g) = B_n(\g) + (1-B_n(\g_+))$ indicate compatibility of $[x]_n$ with $\g$ or not compatible with $\g_+$.
We then have
\[
  h_n(x) = \frac{\mu [x]_n}{\nu [x]_n}\propto \int B_n(\g) \cdot 2^{R_n(\g)} \; d\tilde \nu_n (\gamma)
\]

Note that
$$ R_n(\g) \le Q_n(\g_-,\e) $$
where $Q_n$ denotes the number of edges in $\e$ that connects a vertex $j\ge n$ to a
cluster in $\g_-$ with at least one more edge from the cluster to $[0,n-1]$. That is,
$$  Q_n=\# \{ij \in \epsilon \mid  \exists k\, \exists l\, kl\in\e,  i \sim_{\gamma-} k, k>i, j > n\}.$$
Notice that $Q_n$ only depends on $(\gamma_-,\epsilon)$.



\section{Proofs of the main results}\noindent

\begin{lem}
  There exists a continuous eigenfunction $h$ of ${\mathcal L}$, if, 
$$\frac{\mu[x_0,\ldots x_n, \ldots, x_m]}{\nu[x_0,\ldots, x_n, \ldots, x_m]}=(1+o(1)) \frac{\mu[x_0,\ldots, x_n]}{\nu[x_0, \ldots, x_n]},$$
for all $m\geq n$, as $n\to \infty$.
\end{lem}
\begin{proof}
  Let $\lambda=1$. By the assumption, the measure $\mu= \h \nu$ is translation
  invariant, i.e., $\mu\circ T^{-1}=\mu$. Hence we have $(h\nu)\circ
  T^{-1}=h\nu$ and it suffices to show that $(h\nu)\circ T^{-1}=({\mathcal
    L}h)\nu$.

  Let $A$ be any Borel subset of $X_+$. Then
  $$(h\nu)\circ T^{-1} (A)=\int_A \sum_{y: Ty=x} h(y)e^{\phi(y)}\; d\nu(x)=\int_A h(x)\; d\nu(x).$$
\end{proof}

\noindent
We now fix some notation before proceeding to prove that we have a continuous eigenfunction.
\newline

\begin{center}
\begin{tabular}{rp{0.8\textwidth}}
  $d\tilde\eta(\e)$ & The Bernoulli measure $2^{-|\e|}\ltimes d\eta(\e)$ \\
  $B_n$  & The indicator of the event $[x]_n$ compatible with $\g$. \\
  $B_n^+$  & The indicator of the event $[x]_n$ compatible with $\g_+$. \\
  $B'_n(\g)$ & The indicator of the event $[x]_n$ is
               compatible with $\g$ or not compatible with $\g_+$. \\
  $\hat B_n(\g_-,\e)$ & The indicator of the event that there are
                   no two edges from some cluster $C$ in $\g_-$
                   to a pair $i,j\in[0,n)$ having opposite spins,
                   i.e. such that $x_i x_j = -1$. \\
  $\hat X_n(\g_-,\e)$ & The indicator of the event that $Q_n=0$. \\
  $R_n(\g)$ & Correction term so that
              $$ d\mu(\g|B^+_n) = B'_n \cdot 2^{R_n(\g)} \ltimes d\nu(\g_-) \d\tilde\eta(\e) \d\nu(\g_+|B_n) $$ \\
  $Q_{>n}(\g_-,\e)$ & Number of edges in $\e$ from clusters of $\g_-$ to vertices in $(n,\infty)$
                      such that there is at least one more edge in $\e$ from the same cluster to $[0,\infty)$
                      preceding it in some order\\
  $Q(\g_-,\e)$ & Number of edges in $\e$ from clusters of $\g_-$ to vertices in $[0,\infty)$
                 such that there is at least one more edge in $\e$ from the same cluster to $[0,\infty)$
                 preceding it in some order\\
  $X(C)$ & For a cluster $C\subset \g_{-}$ it is the number of edges in $\e$ to $[0,\infty]$. \\
  $i(C)$ & For a cluster $C\subset \g_{-}$ it is the rightmost vertex, i.e.\ $i(C)=\max \{j\in C\}$. \\
  $\lambda(C)$ & The sum $\lambda(C) = \frac{\beta}{2} \sum_{j\in C} \frac 1j$. \\
\end{tabular}
\end{center}


\begin{lem}
The limit $h(x)=\lim_{n\to\infty} h_n(x)$ is continuous where, form $m\ge1$
\(h_m(x) = \dfrac{\mu([x]_m)}{\nu([x]_m)}\).
\end{lem}

\noindent
{\em Proof} \newline

\noindent
Recall that
\begin{equation}\label{eq:3}
  h_m(x) =  \int B'_m 2^{R_m} \d \nu(\g_-)\d\tl\eta(\e) \d \nu(\g_+|B_m^+).
\end{equation}
Since $R_m$ is the number of edges in $\e$ that do not reduce (with respect to
some order) the number of components in $\g\triangleright [0,n]$, it is clear
that
\begin{equation}\label{eq:RleQ}
  R_m \le Q_{>m}
\end{equation}
where \(Q_{>m}\) is the number of edges $ij\in\e$ where $i>n$ and $-j$ belongs
to a cluster $C$ in $\g_-$ that sends at least one more edge to $[0,\infty)$.

For $n\le m$, we have
$$ \hat B_n \hat X_n \le B'_m \le \hat B_n $$
and, on account of \eqref{eq:RleQ}, it follows that
\begin{equation}\label{eq:4}
  \int \hat B_n \hat X_n \d \nu(\g_-) \d\tl\eta(\e)
  \le h_m(x) \le \int \hat B_n 2^{Q_{>n}} \d\nu(\g_-) \d\tl\eta(\e).
\end{equation}
We have used that $\hat B_n$, $\hat X_n$ and $Q_{>m}$ are independent of $\g_+$ and that
$$\int\d\nu(\g_+|B_m^+)=1. $$

Since both $\hat B_n$ and $\hat X_n$ are decreasing in $(\g_-,\e)$ it follows
from the FKG inequality that
\begin{equation}
  \label{eq:5}
  I_n \cdot \int \hat X_n \d\nu(\g_-) \d\tl\eta(\e)
  \le h_m(x)
  \le I_n \cdot \int 2^{Q_{>n}} \d\nu(\g_-) \d\tl\eta(\e).
\end{equation}
where
\begin{equation}
  \label{eq:6}
  I_n = \int \hat B_n \d\nu(\g_-) \d\tl\eta(\e).
\end{equation}

We prove the following lemma.
\begin{lemma}\label{lem:qn}
  The integral
  \[
    \int 2^{Q_{>n}} \d\nu(\g_-)\, \d\tilde\eta(\e) = 1+\ordo{1}.
  \]
  as $n\to\infty$.
\end{lemma}

From \eqref{eq:5} and this lemma we deduce that
\begin{equation}
  \label{eq:2}
  h_m(x) = (1+\ordo1) I_n = h_n(x) \cdot (1+\ordo1)
\end{equation}
if $m\ge n$ as $n\to\infty$. It follows that $\log h_n(x)$ is a Cauchy sequence
and hence that the limit $h(x)$ is continuous.


\subsubsection*{Proof of Lemma~\ref{lem:qn}}


We condition on a fixed graph $\g_-$ with distribution $\nu_-$. Let $C$ be a cluster of $\g_-$.
Note that
$$
Q=(X(C_1) -1)_{+} +(X(C_2)-1)_{+} + \ldots
$$
where $X(C_i)$, is a sum of independent Bernoulli variables
$$
X(C) = \sum_{-j\in C} \sum_{i=0}^\infty \e_{ji}
$$
where
$$
\P(\e_{ij}=1) = \frac{1- \exp\{-\frac \beta{(i+j)^2}\}}{2}
$$
It follows that we can approximate $X(C)$ with a Poisson variable
$\tl X(C) \sim \opn{Po}(\lambda(C))$ with
$$
\lambda(C) = \frac{\beta}{2} \sum_{j\in C} \frac 1j
\approx \frac{\beta}{2} \sum_{j\in C} \sum_{i=0}^\infty \P(\e_{ij}=1).
$$
Note that
\begin{equation}
  \label{eq:lambdabound}
    \lambda(C) \leq \log \left(1+\frac{|C|}{i(C)}\right)
\end{equation}
where $-i(C)=\max C$.

Order the clusters of $\g_-$ as $C_1,C_2,\dots$ etc. so that
$i(C_1)<i(C_2)<\dots$. For each clister $C_i$ we can from stochastic dominance
construct a random cluster $\tl C_i$ such that (i) $C_i\subset \tl C_i$ and (ii)
$i(\tl C_i)=i(C_i)$. We can further assume that the $\tl C_i$s are \emph{independent}.

Let now
\[
  \tl Q = \sum_{C_i} (\tl X(\tl C_i) - 1)_+.
\]
where $\P(\tl X(\tl C)| \tl C) = \opn{Po}(\lambda(\tl C))$ which stochastically
dominates $X(C)$. Note that for a poissonvariable $X\sim\opn{Po}(\lambda)$ we have
\[
  \E(2^{(X-1)_+}) = \frac{\exp(\lambda(e^{\ln 2}-1)) + e^{\lambda}}{2} = \cosh(\lambda)
\]

We then have
$$
A := \E(2^Q | \gamma_- )\leq \prod \cosh (\lambda(C_i))\leq \prod \cosh (\lambda (\tilde C_i)).
$$
We obtain ($i(C_k)\leq k$)
$$E(A)\leq \prod_{k=1}^\infty \left(1+\frac{E(|\tilde C_k|^2)}{k^2}\right)\cdot K,$$
where $K$ is a constant.


If $|C_k|\leq 0.1 k$, then
$$\cosh (\lambda (\tilde C_k))\leq 1+\left( \frac{|\tilde C_k|}{i(\tilde C_k)}\right)^2,$$
so
$$E[2^Q]\leq \prod_{k=1}^\infty E \left(\cosh \left(\log \left(1+\frac{|\tilde C_k |}{k} \right) \right) \right).$$




We have (with $\tilde C_k$ independent)
$$E\left(\frac{1}{2}(1+\frac{|\tilde C_k|}{k}+\frac{1}{1+\frac{|\tilde C_k|}{k}}) \right)$$
$$=E\left(\frac{1}{2}(1+\frac{|\tilde C_k|^2}{k^2}-\frac{|\tilde C_k|^3}{k^3}+\cdots )\right).$$




\begin{thebibliography}{999}

\bibitem{ACCN} M.\ Aizenman, J.\ Chayes, L.\ Chayes and C.\ Newman,
  Discontinuity of the magnetization in the one-dimensional
  $1/|x-y|^2$ Ising and Potts models, {\em J.\  Statist.\ Phys.\  } {\bf
    50} (1988), 1--40.
    
 \bibitem{berger} N.\ Berger, C.\ Hoffman and V.\ Sidoravicius,
  Nonuniqueness for specifications in $l^{2+\epsilon}$,
  {\em Ergodic Theory \& Dynam.\ Systems} {\bf 38}
  (2018), no.\ 4, 1342--1352.   

\bibitem{berbee87} H. Berbee, Chains with Infinite Connections:
  Uniqueness and Markov Representation, {\em Probab.\ Theory Related
    Fields} {\bf 76} (1987), 243--253.
 
\bibitem{berbee89} H. Berbee, Uniqueness of Gibbs measures and
  absorption probabilities, {\em Ann.\ Probab.\ } {\bf 17} (1989),
  no.\ 4, 1416--1431.
 
\bibitem{berger2} 
N.\ Berger, D.\ Conache, A.\ Johansson, and A.\ \"Oberg, Doeblin measures --- uniqueness and mixing properties,
preprint. 
 
\bibitem{lop1} L.\ Cioletti and A.\ Lopes, Interactions,
  Specifications, DLR probabilities and the Ruelle Operator in the
  One-Dimensional Lattice, preprint, arXiv:1404.3232.
 
\bibitem{lop3} L.\ Cioletti and A.\ Lopes, Ruelle Operator for
  Continuous Potentials and DLR-Gibbs Measures, preprint,
  arXiv:1608.03881v1.
  
\bibitem{doeblin} W.\ Doeblin and R.\ Fortet, Sur des cha{\^i}nes
  {\`a} liaisons compl{\`e}tes, {\em Bull.\ Soc.\ Math.\ France} {\bf
    65} (1937), 132--148.
  
\bibitem{duminil}   
H.\ Duminil-Copin, C.\ Garban, and V.\ Tassion, Long-range models in 1D revisited, preprint.

\bibitem{dyson} F.J.\ Dyson, Non-existence of spontaneous
  magnetisation in a one-dimensional Ising ferromagnet, {\em Commun.\
    Math.\ Phys.\ } {\bf 12} (1969), no.\ 3, 212--215.  

 
\bibitem{FS} J.\ Fr\"ohlich and T.\ Spencer, The phase transition in
  the one-dimensional Ising Model with $1/r^2$ interaction energy,
  {\em Comm.\ Math.\ Phys.\ } {\bf 4} (1982), no.\ 1, 87--101.

\bibitem{johob} A.\ Johansson and A. \"Oberg, Square summability of
  variations of $g$-functions and uniqueness of $g$-measures, {\em
    Math.\ Res.\ Lett.\ } {\bf 10} (2003), no.\ 5-6, 587--601.
    
\bibitem{johob2} A.\ Johansson, A.\ \"Oberg and M.\ Pollicott,
  Countable state shifts and uniqueness of $g$-measures, {\em Amer.\
    J.\ Math.\ } {\bf 129} (2007), no.\ 6, 1501--1511.

\bibitem{johob3} A.\ Johansson, A.\ \"Oberg and M.\ Pollicott, 
Unique Bernoulli $g$-measures, {\em JEMS} {\bf 14} (2012), 1599--1615.

\bibitem{johob4} A.\ Johansson, A.\ \"Oberg and M.\ Pollicott, 
Phase transitions in long-range Ising models and an optimal 
condition for factors of $g$-measures, {\em Ergodic Theory \& Dynam.\ Systems}
{\bf 39} (2019), no.\ 5, 1317--1330.

  
\bibitem{keane} M.\ Keane, Strongly mixing $g$-measures, {\em Invent.\
    Math.\ } {\bf 16} (1972), 309--324.

\bibitem{sin}
Ya.G.\ Sinai, Gibbs measures in ergodic theory, {\em Russian Mathematical Surveys} {\bf 27}(4) (1972), 21--69. 

\bibitem{walters1}
P.\ Walters, Ruelle's operator theorem and $g$-measures, {\em Trans.\ Amer.\ Math.\ Soc.\ } {\bf 214} (1975), 375--387.

\bibitem{walters3}
P.\ Walters, Convergence of the Ruelle Operator, {\em Trans.\ Amer.\ Math.\ Soc.\ } {\bf 353} (2000), 
no.\ 1, 327--347.
\end{thebibliography}


\noindent
Anders Johansson, Department of Mathematics, University of G\"avle,
801 76 G\"avle, Sweden. Email-address: ajj@hig.se\newline

\noindent
Anders \"Oberg, Department of Mathematics, Uppsala University, P.O.\
Box 480, 751 06 Uppsala, Sweden. E-mail-address:
anders@math.uu.se\newline

\noindent
Mark Pollicott, Mathematics Institute, University of Warwick,
Coventry, CV4 7AL, UK. Email-address: mpollic@maths.warwick.ac.uk\newline

\end{document}
