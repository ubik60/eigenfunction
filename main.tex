\documentclass[11pt, a4paper]{amsart}
\usepackage[textsize=footnotesize,textwidth=3cm]{todonotes}
\usepackage[parfill]{parskip} % Begin paragraphs with an empty line rather than an indent


\usepackage{amsthm}
\newtheorem{thm}{Theorem}
\newtheorem{theorem}[thm]{Theorem}
\newtheorem{conjecture}[thm]{Conjecture}
\newtheorem{cor}[thm]{Corollary}
\newtheorem{corollary}[thm]{Corollary}
\newtheorem{lem}[thm]{Lemma}
\newtheorem{lemma}[thm]{Lemma}
\newtheorem{prop}[thm]{Proposition}
\newtheorem{proposition}[thm]{Proposition}

\newtheorem{axiom}{Axiom}
\newtheorem{claim}{Claim}
\theoremstyle{definition}
\newtheorem{defn}{Definition}
\newtheorem{definition}{Definition}
\newtheorem{ex}{Example}
\newtheorem{example}{Example}
\theoremstyle{remark}
\newtheorem{notation}{Notation}
\newtheorem{remark}{Remark}
\newtheorem{rem}{Remark}

\PassOptionsToPackage{override}{xcolor} % Beamer option
\usepackage[utf8]{inputenc}
\usepackage{hyperref}
\usepackage{amsfonts,amssymb,amsmath}
\usepackage{dsfont}
\usepackage[mathscr]{euscript}
\usepackage{enumerate,xspace,threeparttable}
\usepackage{graphicx}
\usepackage{verbatim}

%\providecommand{\ie}{i.e.\ }
\providecommand{\qr}{\eqref}
\renewcommand{\d}{\,d}
\providecommand{\dd}[2]{\dfrac{d#1}{d#2}}
\providecommand{\qtext}[1]{\quad\text{#1 }\quad}

\providecommand{\RR}{\mathbb{R}}
\providecommand{\CC}{\mathbb{C}}
\providecommand{\ZZ}{\mathbb{Z}}
\providecommand{\QQ}{\mathbb{Q}}
\providecommand{\NN}{\mathbb{N}}
\providecommand{\CF}{\mathscr{F}}
\providecommand{\CB}{\mathscr{B}}
\providecommand{\CA}{\mathscr{A}}
\providecommand{\CM}{\mathscr{M}}
\providecommand{\CT}{\mathscr{T}}

\providecommand{\mfrak}{\mathfrak}
\providecommand{\mscr}{\mathscr}
\providecommand{\mc}{\mathcal}
\providecommand{\mb}{\mathbf}
\providecommand{\bs}{\boldsymbol}
\providecommand{\mbb}{\mathbb}
\providecommand{\ms}{\mathsf}
\providecommand{\vv}[2]{\ensuremath{\overrightarrow{#1#2}}} % vector
\providecommand{\opn}{\operatorname}
\providecommand{\ol}{\overline}
\def\ii#1{^{(#1)}}
\def\pp#1{\left(#1\right)}

\renewcommand{\P}{\mathsf{P}}
\renewcommand{\Pr}[1]{\P\pp{#1}}
\providecommand{\E}{\mathsf{E}}
\providecommand{\Ex}[1]{\E\pp{#1}}

\providecommand{\Var}{\opn{Var}}
\providecommand{\var}{\opn{var}} 
\providecommand{\Cov}{\opn{Cov}}
\providecommand{\msf}{\mathsf}
\providecommand{\ett}{\mathsf{1}}

\providecommand{\e}{\epsilon}
\providecommand{\tl}{\tilde}
\providecommand{\g}{\gamma}
\providecommand{\w}{\omega}
\providecommand{\scp}[2]{\ensuremath{\left\langle#1,#2\right\rangle}}

\providecommand{\bmat}[1]{\begin{bmatrix} #1 \end{bmatrix}}
\providecommand{\ds}{\displaystyle}
\renewcommand{\L}{\mathscr{L}}

\def\X{X}
\def\T{T}

\def\scp#1#2{\left\langle #1 , #2 \right\rangle}

\title[Existence of continuous eigenfunction]{Existence of continuous
  eigenfunctions for the Dyson model in the critical phase}

\author{Anders Johansson, Anders \"Oberg, and Mark Pollicott}
\date{}
\begin{document}

\maketitle
\begin{abstract}
  We prove that there exists a continuous eigenfunction of the transfer operator
  defined by potentials for the so-called Dyson model for all inverse critical
  temperatures that are strictly less than the critical inverse temperature.
  This includes all cases when the potential does not have summable variations,
  the classical condition that ensures the existence of a continuous
  eigenfunction of the transfer operator. As a consequence the inverse critical
  temperatures for the one-sided and the two-sided models are the same. We use
  the Bernoulli random model for our method of proof, so the results also hold in
  this wider context. We also derive corresponding results for the Random cluster model
  under assumptions of Duminil-Copin, Garban, and Tassion \cite{duminil}.
\end{abstract}
\def\h{h}


\section{Introduction and results}\noindent

\subsection{The problem}
It is well-known \cite{walters1} that there exists a continuous and strictly
positive eigenfunction $h$ for any transfer operator defined on a symbolic shift
space with a finite number of symbols such that the potential has summable
variations. Here we prove the existence of a continuous eigenfunction for the
important special class of Dyson potentials up to the critical phase, when the
potential does not satisfy the condition of summable variations. We stress that
it is the continuity that is the main difficulty. 

More precisely, let $\T$ be the left shift on the space $\X=S^{{\mathbb Z}_+}$,
where $S$ is a finite set. Let $\phi:\X\to \RR$ be a continuous function defined up
to an additive constant. We refer to $\phi$ as the \emph{one-point potential}. A
\emph{Gibbs measure} $\nu\in\CM(\X)$ for $\phi$ is one that minimises the free energy
$P(\nu;\phi)=\nu(\phi)-H(\nu)$, where $H(\nu)$ denotes the entropy
$\lim_{n\to\infty} H(\nu\vert_{\CF_n})/n$ per time unit. This measure $\nu$ can also be
obtained as the Gibbs measure on $\X$ obtained from the full potential
$$ \Phi(x)=\sum_{k=0}^\infty \phi(\T^k x) $$
on $\X$. A Gibbs measure $\nu$ is also obtained as an eigenmeasure of the dual of
the \emph{transfer operator} $\mc L=\mc L_\phi$ defined on continuous functions by
\begin{equation}\label{trans}
  \mc{L} f(x)= \sum_{y\in T^{-1}x} e^{\phi(y)}\, f(y).
\end{equation}
Such an eigenmeasure $\nu$ satisfies $\mc{L}^* \mu=\lambda \nu$, for some maximal positive
eigenvalue $\lambda>0$.


The \emph{equilibrium measure} $\mu$ is a minimiser of $P(\mu;\phi)$ among all
\emph{translation invariant} measures $\mu\in{\CM}_\T(\X)$. Taking the natural
extension of $\mu$, we can also represent $\mu$ as a translation invariant measure
on the \emph{two-sided} space $\ol\X = S^\ZZ$ and we can alternatively construct
the measure $\mu\in\CM_\T(\ol\X)$ as the Gibbs measure to the full two-sided potential
\[
  \ol \Phi(x)=\sum_{k=-\infty}^\infty \phi(\T^k x)
\]
defined on $x\in\ol\X$.
If there exists a continuous eigenfunction $h(x)$ to the transfer operator, the
measure $\mu$ can be recovered as the Doeblin measure \cite{berger2} (a.k.a.\
$g$-measure in the terminology of Keane \cite{keane}) corresponding to the
\emph{Doeblin function} $g(x)$ where
\begin{equation}\label{g}
  g(x)= \frac{h(x) e^{\phi(y)}}{\lambda h(\T x)}.
\end{equation}

The existence of a continuous eigenfunction $h$ to $\mc{L}$, such that
$\mc{L}h=\lambda h$, is however not automatic in same way the existence of the
eigenmeasure $\nu$ is. If one assumes summable variations of $\phi$,
\begin{equation}\label{sum}
  \sum_{n=1}^\infty \var_n (\phi)<\infty,
\end{equation}
where $\var_n(\phi)=\sup_{x\sim_n y}|\phi(x)-\phi(y)|$ ($x\sim_n y$ means that $x$ and $y$
coincide in the first $n$ entries), then the existence of a continuous
eigenfunction $h(x)$ follows from a ``cone-argument'' used in {\em
  e.g.}, Walters \cite{walters1}. 
  
In this paper, we use that if there is a translation invariant measure $\mu$ which is
absolutely continuous with respect to the Gibbs measure $\mu$ the Radon--Nikodym
derivative $h(x)=\dfrac{d\mu}{d\nu}$ is an eigenfunction of $\mscr L$.

We limit our study to the Dyson--Ising potential, since we exploit a random cluster formalization of
the problem. Fix $\alpha>1$ and $\beta>0$. Let the
one-point long-range Ising potential $\phi=\phi_{\alpha,\beta}$ be given by
$$\phi(x_0, x_1,\ldots)=x_0\cdot \beta \sum_{j=1}^\infty \frac{x_j}{j^\alpha},$$
and define the one-sided and two-sided Dyson potentials, $\Phi:\X\to \RR$ and
$\bar\Phi:\ol\X\to\RR$ as above. Let $\mu$ and $\nu$ be the Gibbs measures on $\CM(\X)$
and $\CM(\ol\X)$, corresponding to $\Phi$ and $\ol\Phi$, respectively.

\subsection{The result}
For $\alpha>2$ we then have summable variations of the one-sided potential $\Phi$, in fact it is H\"older continuous. We then have 
$$ \mu= h\nu, $$
where $h>0$ is a H\"older continuous eigenfunction. We are interested in the
boundary cases where $1<\alpha\le2$, when there exists a unique equilibrium measure for
$\ol\Phi$ for $\beta<\beta_c$ \cite{ACCN} and multiple equilibrium measures when $\beta>\beta_c$,
where the critical parameter $\beta_c=\beta_c(\alpha)>0$ for $1<\alpha\le2$. We show here that the
same uniqueness properties holds for the ``one-sided'' Gibbs measure $\nu$.

In this case the summable variations condition is not satisfied for neither
$\ol\Phi$ nor $\Phi$; hence we may have multiple eigen-measures for $\mc{L}^*$. In
this context we have $\var_n(\phi)=O(\frac{1}{n})$.

Crucial to our proof of the continuity of an eigenfunction is the existence of a certain dominating random variable whose boundedness in $L^1$ (Lemma \ref{lem:qn}) follows from known estimates of the exponential decay of the upper tail cluster size distribution (Lemma \ref{geometric-bound}). Since this is known better for certain models than others, we also obtain slightly different results for these different models. We obtain the existence of a continuous eigenfunction when $\beta(\alpha)<\beta_c(\alpha)$, whenever $\alpha>1$, for the Bernoulli model ($q=1$ in the random cluster model), but for other models ($q>1$), we obtain these results for some $\beta^*(\alpha) \leq \beta^*(\alpha) <\beta_c (\alpha)$, $\alpha>1$. In all cases we have $\alpha\leq 2$, summable a variations condition on the potential is not satisfied.

Let $\mu$ be the Gibbs equilibrium measure with respect
to the Dyson potential $\phi$ and let $\nu$ be the one-sided Gibbs measure.
Define
$$
h_n(x)=\frac{\mu[x_0,\ldots x_n]}{\nu[x_0,\ldots, x_n]},
$$
and consider the measurable function $h(x)=\limsup_{n\to \infty}h_n(x)$. 

\begin{thm}\label{main} 
If $\alpha>1$ and $\beta(\alpha)$ is sufficiently small, then $h(x)$ is a continuous function on $\X$, $h>0$, and it is also an eigenfunction of $\mc{L}$. In the Bernoulli case we assume only that $\beta(\alpha)<\beta_c$, that is for all subcritical inverse temperatures.
\end{thm}

\subsection{A conjecture}
We conjecture that there exists a continuous eigenfunction for a potential
$\phi$ whenever that potential satisfies Berbee's condition \cite{berbee89}. Berbee proved that equilibrium measures are unique if we have
\begin{equation}\label{berbee}
  \sum_{n=1}^\infty e^{-r_1-r_2-\cdots-r_n}=\infty,
\end{equation}
where $r_n=\var_n \Phi$ or $r_n=\var_n \ol\Phi$, that is for both the one-sided and two-sided models.

In \cite{johob} it was proved that uniqueness $g$-measures follows
whenever
$$ \sum_{n=1}^\infty (\var_n g)^2<\infty, $$
and more recently Berger et al.\ \cite{berger2} proved that for such {\em Doeblin
measures} uniqueness follows if $\var_n \log g <2/\sqrt{n}$ (see also
\cite{johob3} for a slightly stronger condition).

This connects to Dyson's counterexample in \cite{dyson} where it is shown that
there are examples of multiple equilibrium measures for $\phi$ whenever
$$\sum_{n=1}^\infty (\var_n \phi)^{1+\epsilon}<\infty,$$
where $\epsilon>0$.

Dyson's example is for a two-sided model, but we showed in \cite{johob4} that
the inverse critical temperature $\beta_c^+$ satisfies $\beta_c^+\leq 8\beta_c$, where $\beta_c$ is
the inverse critical temperature for the two-sided model, and this show that
Dyson's example of multiple equilibrium measures can be formulated for a
one-sided model as above.

Since we have the ``translation'' via \eqref{g} between the case for general
potentials and for Doeblin measures, we may guess that the existence of a
continuous eigenfunction cannot be moved very far away from the summability of
variations condition for a potential.

\section{The one-dimensional random cluster model and the Ising--Dyson model}

For a finite graph, let $\w(G)$ denote the number of connected components
(``clusters'') in the graph $G$. For simple graphs $G\subset \binom V2$ on an
countably infinite set $V$ of vertices, we consider the number of clusters
$\w(G)$ as a \emph{potential}. This means that the difference
$\Delta\w(G,F) = \w(G)-\w(F)$ is defined for any two graphs $F$ and $G$ that
coincide outside a \emph{finite} subset $\Lambda\subset \binom V2$.

A random graph $G\sim\alpha$ on a set of vertices $V$ is a probability
distribution $\alpha$ on the set $\{0,1\}^{\binom V2}$. The random cluster
models $\mscr R(V,p,q)$ (FK-model \cite{grimmet}), we consider are specified by a set 
of vertices $V$ and a probabilities $p(ij)\in[0,1]$, defined on the set of pairs
$ij\in \binom V2$. The model $\mscr R(V,p,q)$ is a Gibbs distribution on random
graphs, i.e.\ configurations i $\gamma\in\{0,1\}^{\binom V2}$, with 
non-continuous potential
$$
\phi(\gamma) = 
- \log q \cdot\w(\gamma) + 
\sum_{ij}\gamma(ij)\cdot \log p(ij) + (1-\gamma(ij))\cdot\log (1-p(ij)).
$$
The models $\mscr R(V,p,1)=\eta(\binom V2,p)$ 
are referred to as ``Bernoulli percolation''. 
In the one-dimensional Potts--Dyson random cluster model, 
$\mscr R(V,\beta,\alpha,q)$,
we let $V=\ZZ$ (or $V=\ZZ_{\ge0}$ and consider the random cluster model 
$\mscr R(V,p_J,q)$ where 
\begin{equation}\label{eq:Jdef}
  p_J(ij) = 1-e^{-J(ij)} \quad\text{where}\quad 
  J({ij}) = \frac \beta{|i-j|^\alpha}.
\end{equation}
When $q$

Percolation is the event that the random graph contains an infinite cluster. 
It is well-known \cite{ACCN} that for all $\alpha\in(1,2]$
there exists a critical $\beta_c=\beta_c(\alpha)$
such that, percolation does not occur with probability 1 for $0<\beta<\beta_c$, 
while for $\beta\in(\beta_c,\infty)$ there is with probability 1
no infinite cluster.     

We will mainly consider the subcritical case $1<\alpha\le 2$ and $0<\beta <\beta_c(\alpha)$ and use the following lemma. 
\begin{lem}\label{geometric-bound}
 Let $X=X(C_s)$ denote the size of the cluster $C_s$ 
 containing a specified vertex $s\in V$. Then there are constants 
 $K$ and $r$, $0<r<1$, such that $$\P(X\ge m)\le K r^{m}. $$ 
\end{lem}
\begin{proof}
  As surveyed by Panagiotis \cite{pan} (Theorem 1.2.1), we know that it
  holds for Bernoulli percolation ($q=1$) when $0<\beta<\beta_c$. Since $\mscr R(V,p,1)$
  stochastically dominates $\mscr R(V,p,q)$, $q>1$, it also holds for the
  FK-model. Finally, Duminil-Copine (\cite{duminil}) proves that $\beta_c(\alpha,q)$ are
  the same for the case $q=1$ and $q>1$.
\end{proof}

\subsubsection{The extended random cluster model}
The extended \emph{random cluster model} with $\mscr R_e(V,p,q)$ can be obtained
as the joint distribution of the spin sequences $x\in\{1,2,\dots,q\}^{V}$ and a
random graph $\gamma\in\{0,1\}^{\binom V2}$. The distribution of $(x,\gamma)$ is obtained by
first considering the pair chosen independently: The spin sequence
$x\in\{0,1,\dots,q\}^{V}$ according to the uniform Bernoulli measure
$x\sim\eta(V,p=\frac1q)$ on the spin sequences and the random graph $\gamma$ according to
the Bernoulli measure $\eta(\binom V2, p)=\mscr R(V,p,1)$. Then $\mscr R_e(V,p,q)$
is the distribution of $(x,\gamma)$ \emph{conditioned on} $x$ and $\gamma$ being
\emph{compatible}: That is, the event $C(x,\g)$ that no spins $x_i=+1$ and
$x_j=-1$ in $x$ are connected by a path (edge) in $\g$.

Assume $(x,\mu)\sim\mu=\mscr R_e(V,p,q)$. Then marginal distribution $\mu\circ x^{-1}$ of $x$
is the Potts model with interactions given by $J(ij)=\log(1-p(ij))$ and the
marginal distribution $\mu\circ\gamma^{-1}$ of $\gamma$ is the random cluster model
$\mscr R(V,p,q)$ described above.

One obtains the long range Ising model with the Dyson potential $\phi_{\beta,\alpha}(x)$ as
the marginal distribution of $x$ and the random cluster model $\mscr R(V,p,2)$
as the marginal distributions $\g$.

If $\mu$ is a random cluster measure, we have 

\def\cc#1{{\langle #1 \rangle}}

Let $S\subset V$ be a finite subset.
We need to establish the probability $\mu([x]_S)$ of a cylinder 
$[x]_S=\{(y,\g) \mid y\vert_S = x\vert_S\}$ under $\mu=\mscr R_e(V,p,q)$.
Let $B([x]_S,\g)$ be the indicator for the event that $\g$ is compatible with 
the cylinder $[x]_S$. Let also $\w_S(\g)$ denote the number of clusters in $\gamma$ that intersects $S$ and $\bar\w_S(\g)$ the number of components in the graph $\g\triangleright S$
obtained from joining all 
vertices in $S$ in a cluster, e.g. by adding a vertex $s$ connected to the elements of $S$.

In our application, when $S$ are the intervals $[0,n)$ on the integer line,
we write $[x]_n$, $\w_n$, $\bar\w_n$, $\mu_{\bar n}$, etc. 

\begin{lem}\label{probcyl}
Under an extended random cluster model then for any $x\in\X$
\begin{align}
    \mu([x]_S) &=  \int 2^{-\w_S(\g)} B([x]_S,\g) \d\mu(\g) \\
    &= 2^\cc{-\w_S(\g)} \cdot \int B([x]_S,\g) \d \mu_{\bar S}(\g)  
\end{align}
where $\mu_{\bar S}$ denote the random cluster model and 
$$ \cc{-\w_S(\g)} = \log_2 \int 2^{-\w_S(\g)} \d\mu(\g,x) $$
\end{lem}


We now observe that a configuration $\g$ in the two-sided model
$\mscr R(\ZZ,\alpha,\beta)$ difference between the one-sided random cluster model
$\nu = \mscr R(V_+,J)$ and the usual two-sided model $\mu = \mscr R(V,J)$. We will
use that a configuration $\g$ can be factored as $\g = (\g_+, \e, \g_-)$, where
$\g_-$ is the induced graph $\g[V_-]$ on vertices $-j\in V_-=\ZZ_{<0}$ and
$\g_+=\g[V_+]$ is the graph induced on vertices $i\in V_+=\ZZ_{\ge0}$. 

\begin{lem}
\begin{align}
\mu(\g) &= \mu(\g_+) \otimes \tilde\eta(\e) \otimes \mu(\g_-) \\
\mu_{\bar S}(\g) &= 2^{R_S(\g)} \cdot \mu_{\bar S}(\g_+) \otimes \tilde\eta(\e) \otimes \mu(\g_-) 
\end{align}
where 
$$
R_S(\g) = |\e|-\w_{\bar S}(\g)+ 
$$
\end{lem}

The graph
$\e=\g\cap E(V_+,V-)$ consists of edges $ji$, $i\ge0$ and $j\ge 1$, connecting vertices
$-j\in V_-$ with vertices $i\in V_+$. Note that we often use positive indices $i,j$,
$i\ge0$ and $j>0$, as labels for edges in $\e$. Thus $J(ij)=\beta/(i+j)^\alpha$ with this
labelling.

We then have (analogously to Lemma \ref{probcyl})
 \[
    \nu([x]_n) =  \int 2^{-\w_n(\g_+)} B(x_n,\g_+) \d\nu(\g_+).
     \]


Let
$$ \tilde \eta(\epsilon)= 2^{-|\epsilon|} \ltimes \eta (\epsilon). $$
Note that both $\eta(\cdot)$ and $\tilde\eta (\cdot)$ are Bernoulli measures. Let also
$$ d\tilde \nu_n(\gamma) = \frac{d\nu(\gamma_-)\otimes \tilde \eta(\epsilon)\otimes \nu(\gamma_+)}{\nu(B([x]_n,\gamma_+))}. $$

If $\mu$ is a random cluster measure, we can write (with a suitable normalisation to make it a probability measure):
$$\mu= 2^{R(\g)-K} \d\nu(\g_-)\d\tilde \eta(\epsilon) \d(\g_+),$$
where $2^K$ is the normalisation, and $R(\g)$ is the ``co-rank of $\epsilon$ in $\g$", i.e.,
$$R(\g)=|\epsilon|-(\omega(\g \setminus \epsilon)-\omega(\g)).$$

Note that
$$ R(\g) \le Q(\g_-,\e) =\sum \left(d(c,\epsilon)-1\right)_+$$
Notice that $Q$ only depends on $(\gamma_-,\epsilon)$. 

We have
\begin{align}
  h_n(x) &= 
  \frac{\int B([x]_n, \g) \cdot 2^{R(\g)-K+\omega_n(\g_+)-\omega_n(\g_-)}
  \d\nu(\g_+)\d \tilde \eta(\e)\d\nu(\g_-)}
  {\int B([x]_n,\g_+) \cdot 2^{-\omega_n(\g_+)} d\nu(\g_+)},
\end{align}
where 
$\omega_n(\g)$ is the number of clusters in $\g$ intersecting $[0,n)$. We define 
$$R_n=R(\g)-K+\omega_n(\g_+)-\omega_n(\g_-),$$ and 
$$B_n'(x,\g)=\frac{B([x]_n, \g)}{B([x]_n,\g_+)},$$
and write
$$ h_n(x) = \int B_n'(x,\g) 2^{R_n(\g)}\d\tilde \nu_n(\g_+)\d \tilde \eta(\e)\d\nu(\g_-),$$
where $\nu_n$ is normalised. We now study the sequence of functions 
$$g_n(\epsilon, \g_-)=\int B_n'(x,\g) 2^{R_n(\g)}\d\tilde \nu_n(\g_+).$$

Crucial to our proof of the continuity of $h$ is the $L^1$-bound of $2^Q$, in order to prove the continuity of $h$.

\section{Proof of Theorem 1}\noindent

\begin{lem}
  If $\mu\in\CM_\T(\X)$ is a translation invariant measure which is absolutely
  continuous with the Gibbs measure $\nu$ for $\phi$ then the Radon--Nikodym
  derivative $h(x)=\dfrac{d\mu}{d\nu}(x)$ is an eigenfunction to the transfer
  operator $\mscr L =\mscr L_\phi$.
\end{lem}
\begin{proof}
  We can assume $\lambda=1$. By assumption, the measure $\mu= \h \nu$ is translation
  invariant, i.e., $\mu\circ T^{-1}=\mu$. Hence we have $(h\nu)\circ T^{-1}=h\nu$ and it
  suffices to show that $(h\nu) \circ T^{-1}=({\mathcal L}h)\nu$.

  Let $A$ be any Borel subset of $\X$. Then
  $$(h\nu)\circ T^{-1} (A)=\int_A \sum_{y: Ty=x} h(y)e^{\phi(y)}\; d\nu(x)=\int_A h(x)\; d\nu(x).$$
\end{proof}

We need to prove that the $h_n(x)$ converge to a continuous function $h(x)$ as
$n\to\infty$. It is enough to show that
$$ \lim_{m\geq n\to \infty}\sup_{x} |h_m(x)-h_n(x)|=0.$$
We are then certain to have a continuous eigenfunction $h$ that is strictly positive, since if it had zeroes it would be identically zero, since if $x_0$ is a zero of $h$, then we have (with the eigenvalue $\lambda $ normalised to 1):
$$0= h(x_0)=\sum_{y\in T^{-n}x_0} e^{\Phi(y)} h(y).$$
That $h$ is not identically zero follows because it must integrate to $1$ with respect to $\nu$, since $\int h_n \d\nu=1$ for all $n$.

Recall that
\begin{equation}\label{eq:3}
  h_n(x) =  \int B'_n 2^{R_n} \d \nu(\g_-)\d\tl\eta(\e) \d \nu_n(\g_+|B_n^+).
\end{equation}
Let 
$$g_n(x,\gamma_-,\epsilon)=\int B_n' (\gamma_+,\epsilon, \gamma_-) 2^{R_n(\gamma_+,\epsilon,\gamma_-)} \,d\nu_n(\g_+|B_n^+),$$
so that 
$$h_n(x)= \int g_n(x,\gamma_-,\epsilon) \d \nu_n(\g_-)\d\tl\eta(\e).$$
We also notice that since $R_n$ is the number of edges in $\e$ that do not reduce (with respect to
some order) the number of components in $\g\triangleright [0,n]$, it is clear
that
\begin{equation}\label{eq:RleQ}
  R_n \le Q_{n} 
\end{equation}
where \(Q_{n}\) is the number of edges $ij\in\e$ where $i>n$ and $-j$ belongs
to a cluster $C$ in $\g_-$ that sends at least one more edge to $[0,\infty)$.

We notice that  for $n>N$, where is $N=N(\epsilon, \gamma_-)$ is some random variable, $ B_n' (\gamma_+,\epsilon, \gamma_-) 2^{R_n(\gamma_+,\epsilon,\gamma_-)} $ is constant (in $n$). For $n\geq N$, we have $R_n(\gamma_+,\epsilon,\gamma_-)=Q(\epsilon, \gamma_-)=\lim_n Q_n(\epsilon, \gamma_-)$, and $B_n'=B_N'$.

We have that $N<\infty$ a.s., since $N\leq Q\leq 2^Q$ and we have the following lemma.
\begin{lemma}\label{lem:qn}
  The integral
  $$
    \int 2^{Q(\gamma_-,\epsilon)} \d\nu(\g_-)\, \d\tilde\eta(\e) <\infty.
  $$
\end{lemma}
We now conclude from dominated convergence (using $2^Q$ as the dominating function) that since $g_n(\epsilon, \gamma_-)-g_m(\epsilon, \gamma_-)=0$, if $N\leq n\leq m$, and otherwise less than or equal to $2^Q$, we have 
$$ \lim_{m\geq n\to \infty}\sup_{x} |h_m(x)-h_n(x)|\leq \lim_{m\geq n\to \infty} \int |g_n(x,\gamma_-,\epsilon)-g_m(x,\gamma_-,\epsilon)   |\; \d\nu(\g_-)\, \d\tilde\eta(\e)=0.$$


\subsubsection*{Proof of Lemma~\ref{lem:qn}}

We condition on a fixed graph $\g_-$ with distribution $\nu_-$. Let $C$ be a
cluster of $\g_-$. Note that
$$ Q=(X(C_1) -1)_{+} +(X(C_2)-1)_{+} + \ldots $$
where $X(C)$, is a sum of independent Bernoulli variables
$$ X(C) = \sum_{-j\in C} \sum_{i=0}^\infty \e_{ji} $$
where
$$
\P(\e_{ij}=1) = \frac{1- \exp\{-\frac \beta{(i+j)^2}\}}{2}
$$
It follows that we can approximate $X(C)$ with a Poisson variable (***)
$\tl X(C) \sim \opn{Po}(\lambda(C))$ with
$$
\lambda(C) = \frac{\beta}{2} \sum_{j\in C} \frac 1j
\approx \frac{\beta}{2} \sum_{j\in C} \sum_{i=0}^\infty \P(\e_{ij}=1).
$$
Note that
\begin{equation}
  \label{eq:lambdabound}
    \lambda(C) \leq \log \left(1+\frac{|C|}{i(C)}\right)
\end{equation}
where $-i(C)=\max C$.

Order the clusters of $\g_-$ as $C_1,C_2,\dots$ etc. so that
$i(C_1)<i(C_2)<\dots$. For each cluster $C_i$ we can from stochastic dominance
construct a random cluster $\tl C_i$ such that (i) $C_i \subset \tl C_i$ and (ii)
$i(\tl C_i)=i(C_i)$. We can further assume that the $\tl C_i$s are
\emph{independent} with the same distribution.

Let now
\[
  \tl Q = \sum_{C_i} (\tl X(\tl C_i) - 1)_+.
\]
where $\P(\tl X(\tl C)| \tl C) = \opn{Po}(\lambda(\tl C))$ which stochastically
dominates $X(C)$. For a poisson variable $X\sim\opn{Po}(\lambda)$ we have
\[
  \E(2^{(X-1)_+}) = \frac{\exp(\lambda(e^{\ln 2}-1)) + e^{\lambda}}{2} = \cosh(\lambda)
\]

We then have
$$
\E(2^Q | \gamma_- ) \leq \prod \cosh (\lambda(C_i))\leq \prod \cosh (\lambda (\tilde C_i)).
$$
We obtain, since $i(C_k)\geq k$ and \qr{eq:lambdabound} and the independence of
$\tl C_k$, that
\begin{align}
  E(2^Q) &\leq \prod_{k=1}^\infty \E\left(\frac{1}{2}\left(1+\frac{Y}{k}+\frac{1}{1+\frac{Y}{k}}\right) \right) \\
         &\leq \prod_{k=1}^\infty \E\left(1+\frac{Y^2}{k^2} + \frac{Y^3}{k^3} + \dots \right).
\end{align}
where $Y$ has the common distribution of $|C_k|$. It is easy to see that this is
less than $\infty$ on account of Lemma~\ref{geometric-bound}, which states that the
distribution $Y$ has an exponentially decreasing bound for the upper tail.


\begin{thebibliography}{999}

\bibitem{ACCN} M.\ Aizenman, J.\ Chayes, L.\ Chayes and C.\ Newman,
  Discontinuity of the magnetization in the one-dimensional
  $1/|x-y|^2$ Ising and Potts models, {\em J.\  Statist.\ Phys.\  } {\bf
    50} (1988), 1--40.
    
 \bibitem{berger} N.\ Berger, C.\ Hoffman and V.\ Sidoravicius,
  Nonuniqueness for specifications in $l^{2+\epsilon}$,
  {\em Ergodic Theory \& Dynam.\ Systems} {\bf 38}
  (2018), no.\ 4, 1342--1352.   

\bibitem{berbee87} H. Berbee, Chains with Infinite Connections:
  Uniqueness and Markov Representation, {\em Probab.\ Theory Related
    Fields} {\bf 76} (1987), 243--253.
 
\bibitem{berbee89} H. Berbee, Uniqueness of Gibbs measures and
  absorption probabilities, {\em Ann.\ Probab.\ } {\bf 17} (1989),
  no.\ 4, 1416--1431.
 
\bibitem{berger2} 
N.\ Berger, D.\ Conache, A.\ Johansson, and A.\ \"Oberg, Doeblin measures --- uniqueness and mixing properties,
preprint. 
 
\bibitem{lop1} L.\ Cioletti and A.\ Lopes, Interactions,
  Specifications, DLR probabilities and the Ruelle Operator in the
  One-Dimensional Lattice, preprint, arXiv:1404.3232.
 
\bibitem{lop3} L.\ Cioletti and A.\ Lopes, Ruelle Operator for
  Continuous Potentials and DLR-Gibbs Measures, preprint,
  arXiv:1608.03881v1.
  
\bibitem{doeblin} W.\ Doeblin and R.\ Fortet, Sur des cha{\^i}nes
  {\`a} liaisons compl{\`e}tes, {\em Bull.\ Soc.\ Math.\ France} {\bf
    65} (1937), 132--148.
  
\bibitem{duminil}   
H.\ Duminil-Copin, C.\ Garban, and V.\ Tassion, Long-range models in 1D revisited, preprint.

\bibitem{dyson} F.J.\ Dyson, Non-existence of spontaneous
  magnetisation in a one-dimensional Ising ferromagnet, {\em Commun.\
    Math.\ Phys.\ } {\bf 12} (1969), no.\ 3, 212--215.  

 
\bibitem{FS} J.\ Fr\"ohlich and T.\ Spencer, The phase transition in
  the one-dimensional Ising Model with $1/r^2$ interaction energy,
  {\em Comm.\ Math.\ Phys.\ } {\bf 4} (1982), no.\ 1, 87--101.

\bibitem{johob} A.\ Johansson and A. \"Oberg, Square summability of
  variations of $g$-functions and uniqueness of $g$-measures, {\em
    Math.\ Res.\ Lett.\ } {\bf 10} (2003), no.\ 5-6, 587--601.
    
\bibitem{johob2} A.\ Johansson, A.\ \"Oberg and M.\ Pollicott,
  Countable state shifts and uniqueness of $g$-measures, {\em Amer.\
    J.\ Math.\ } {\bf 129} (2007), no.\ 6, 1501--1511.

\bibitem{johob3} A.\ Johansson, A.\ \"Oberg and M.\ Pollicott, 
Unique Bernoulli $g$-measures, {\em JEMS} {\bf 14} (2012), 1599--1615.

\bibitem{johob4} A.\ Johansson, A.\ \"Oberg and M.\ Pollicott, 
Phase transitions in long-range Ising models and an optimal 
condition for factors of $g$-measures, {\em Ergodic Theory \& Dynam.\ Systems}
{\bf 39} (2019), no.\ 5, 1317--1330.

  
\bibitem{keane} M.\ Keane, Strongly mixing $g$-measures, {\em Invent.\
    Math.\ } {\bf 16} (1972), 309--324.
    
\bibitem{pan} 
C.\ Panagiotos, {\em Interface theory and Percolation}, PhD Thesis, University of Warwick 2020.    

\bibitem{sin}
Ya.G.\ Sinai, Gibbs measures in ergodic theory, {\em Russian Mathematical Surveys} {\bf 27}(4) (1972), 21--69. 

\bibitem{walters1}
P.\ Walters, Ruelle's operator theorem and $g$-measures, {\em Trans.\ Amer.\ Math.\ Soc.\ } {\bf 214} (1975), 375--387.

\bibitem{walters3}
P.\ Walters, Convergence of the Ruelle Operator, {\em Trans.\ Amer.\ Math.\ Soc.\ } {\bf 353} (2000), 
no.\ 1, 327--347.
\end{thebibliography}


\noindent
Anders Johansson, Department of Mathematics, University of G\"avle,
801 76 G\"avle, Sweden. Email-address: ajj@hig.se\newline

\noindent
Anders \"Oberg, Department of Mathematics, Uppsala University, P.O.\
Box 480, 751 06 Uppsala, Sweden. E-mail-address:
anders@math.uu.se\newline

\noindent
Mark Pollicott, Mathematics Institute, University of Warwick,
Coventry, CV4 7AL, UK. Email-address: mpollic@maths.warwick.ac.uk

\end{document}



